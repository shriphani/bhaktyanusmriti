\documentclass{book}

\usepackage{polyglossia}
\setdefaultlanguage{sanskrit}
\setotherlanguage{english}

\newfontfamily\sanskritfont[Script=Devanagari]{Siddhanta}
\newfontfamily\englishfont[Script=Latin]{Palatino}


\usepackage[margin=2in]{geometry}
\usepackage{xltxtra}
\usepackage{fontspec}
\usepackage{varwidth}
\usepackage{titlesec}
\usepackage{titling}
\usepackage{tcolorbox} % For creating a box around the verse
\usepackage{setspace} % For adjusting line spacing
\usepackage{adjustbox}
\usepackage{multicol} % For creating multi-column layout


% Customize the title
\pretitle{\begin{center}\Huge\bfseries}
\posttitle{\par\end{center}\vskip 0.5em}
\preauthor{\begin{center}\Large}
\postauthor{\par\end{center}}
\predate{\begin{center}\large}
\postdate{\par\end{center}}

% Customize the verse environment
\newenvironment{Verse}[1][]
  {\bigskip\noindent\begin{center}\begin{adjustbox}{valign=M}\begin{tcolorbox}[colframe=black,colback=white,boxrule=0.5mm,arc=4mm,auto outer arc,left=2mm,right=2mm,top=2mm,bottom=2mm,halign=center]
  \LARGE\bfseries#1\par\varwidth{\linewidth}\centering}
  {\endvarwidth\end{tcolorbox}\end{adjustbox}\end{center}\bigskip}

\title{भक्त्यानुस्मृति}
\author{श्रीफणी पालाकोेडेटी}
\date{}

\begin{document}

\maketitle

\begin{Verse}
    \setstretch{1.5} % Adjust line spacing within the verse
    शुक्लाम्बरधरं विष्णुं शशिवर्णं चतुर्भुजम् ।\\
    प्रसन्नवदनं ध्यायेत् सर्वविघ्नोपशान्तये ॥
\end{Verse}

\vspace{-2em} % Add some vertical space before the citation
\begin{center}
    अवन्त्यखण्डे स्कन्दपुराणे
\end{center}
\vspace{2em}

\begin{multicols}{2}
    \setlength{\columnseprule}{0.4pt}
    \begin{itemize}
        \item शुक्लाम्बरधरं = शुक्ल + अम्बर + धरं
        \item शशिवर्णं = शशि + वर्णं
        \item चतुर्भुजम् = चतुर् + भुजम्
        \item प्रसन्नवदनं = प्रसन्न + वदनं
        \item सर्वविघ्नोपशान्तये = सर्व + विघ्न + उपशान्तये
    \end{itemize}

    \columnbreak
    
    \textenglish{
        \noindent Sri Vishnu - who wears white, is of a moon-like complexion, is four-armed,
        and has a pleasant visage - we medidate on Him to ward off all obstacles.
    }

\end{multicols}

\clearpage

\begin{Verse}
    \setstretch{1.5} % Adjust line spacing within the verse
    अगजानन पद्मार्कं गजाननं अहर्निशम् ।\\
    अनेकदंतं भक्तानां एकदन्तं उपास्महे ॥
\end{Verse}

\begin{multicols}{2}
    \setlength{\columnseprule}{0.4pt}
    \begin{itemize}
        \item अगजानन = अगजा + आनन
        \item पद्मार्क = पद्म + अर्क
        \item गजानन = गजा + आनन
        \item अहर्निशा = अहः + निशा
        \item अनेकदंत = अनेक + दम्  + तम्
        \item एकदन्त = एक + दन्त
    \end{itemize}

    \columnbreak
    
    \vspace{10pt}
    \textenglish{
        \noindent We meditate on Lord Ganesha,\\
        who has the face of an elephant,\\
        whose presence always causes Devi Parvati's face to bloom like a lotus,\\
        who grants the many wishes and desires of His devotees,\\
        and who possesses one tusk.
    }

\end{multicols}


\end{document}