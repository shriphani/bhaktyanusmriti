\documentclass{book}

\usepackage{polyglossia}
\setdefaultlanguage{sanskrit}
\setotherlanguage{english}

\newfontfamily\sanskritfont[Script=Devanagari]{Siddhanta}
\newfontfamily\englishfont[Script=Latin]{Palatino}


\usepackage[margin=2in]{geometry}
\usepackage{xltxtra}
\usepackage{fontspec}
\usepackage{varwidth}
\usepackage{titlesec}
\usepackage{titling}
\usepackage{tcolorbox} % For creating a box around the verse
\usepackage{setspace} % For adjusting line spacing
\usepackage{adjustbox}
\usepackage{multicol} % For creating multi-column layout


% Customize the title
\pretitle{\begin{center}\Huge\bfseries}
\posttitle{\par\end{center}\vskip 0.5em}
\preauthor{\begin{center}\Large}
\postauthor{\par\end{center}}
\predate{\begin{center}\large}
\postdate{\par\end{center}}

% Customize the verse environment
\newenvironment{Verse}[1][]
  {\bigskip\noindent\begin{center}\begin{adjustbox}{valign=M}\begin{tcolorbox}[colframe=black,colback=white,boxrule=0.5mm,arc=4mm,auto outer arc,left=2mm,right=2mm,top=2mm,bottom=2mm,halign=center]
  \LARGE\bfseries#1\par\varwidth{\linewidth}\centering}
  {\endvarwidth\end{tcolorbox}\end{adjustbox}\end{center}\bigskip}

\title{भक्त्यानुस्मृति}
\author{श्रीफणी पालाकोेडेटी}
\date{}

\begin{document}

\maketitle

\tableofcontents

\chapter{गणेश श्लोकाः}

\begin{Verse}
    \setstretch{1.5} % Adjust line spacing within the verse
    शुक्लाम्बरधरं विष्णुं शशिवर्णं चतुर्भुजम् ।\\
    प्रसन्नवदनं ध्यायेत् सर्वविघ्नोपशान्तये ॥
\end{Verse}

\vspace{-2em} % Add some vertical space before the citation
\begin{center}
    अवन्त्यखण्डे स्कन्दपुराणे
\end{center}
\vspace{2em}

\begin{multicols}{2}
    \setlength{\columnseprule}{0.4pt}
    \begin{itemize}
        \item शुक्लाम्बरधरं = शुक्ल + अम्बर + धरं
        \item शशिवर्णं = शशि + वर्णं
        \item चतुर्भुजम् = चतुर् + भुजम्
        \item प्रसन्नवदनं = प्रसन्न + वदनं
        \item सर्वविघ्नोपशान्तये = सर्व + विघ्न + उपशान्तये
    \end{itemize}

    \columnbreak
    
    \textenglish{
        \noindent Sri Vishnu - who wears white, is of a moon-like complexion, is four-armed,
        and has a pleasant visage - we medidate on Him to ward off all obstacles.
    }

\end{multicols}

\clearpage

\begin{Verse}
    \setstretch{1.5} % Adjust line spacing within the verse
    अगजानन पद्मार्कं गजाननं अहर्निशम् ।\\
    अनेकदंतं भक्तानां एकदन्तं उपास्महे ॥
\end{Verse}

\begin{multicols}{2}
    \setlength{\columnseprule}{0.4pt}
    \begin{itemize}
        \item अगजानन = अगजा + आनन
        \item पद्मार्क = पद्म + अर्क
        \item गजानन = गजा + आनन
        \item अहर्निशा = अहः + निशा
        \item अनेकदंत = अनेक + दम्  + तम्
        \item एकदन्त = एक + दन्त
    \end{itemize}

    \columnbreak
    
    \vspace{10pt}
    \textenglish{
        \noindent We meditate on Lord Ganesha,\\
        who has the face of an elephant,\\
        whose presence always causes Devi Parvati's face to bloom like a lotus,\\
        who grants the many wishes and desires of His devotees,\\
        and who possesses one tusk.
    }

\end{multicols}

\clearpage

\section{संकटनाशन गणेश स्तोत्रम्}

\begin{Verse}
    \setstretch{1.5} % Adjust line spacing within the verse
    नारद उवाच ।\\
    प्रणम्य शिरसा देवं गौरीपुत्रं विनायकम् ।\\
    भक्तावासं स्मरेन्नित्यमायुष्कामार्थसिद्धये ॥१॥
\end{Verse}


\begin{multicols}{2}
    \setlength{\columnseprule}{0.4pt}
    \begin{itemize}
        \item गौरीपुत्रं = गौरी + पुत्रं
        \item भक्तावासं = भक्त + आवासं
        \item स्मरेन्नित्यमायुष्कामार्थसिद्धये = स्मरेत् + नित्यम् + आयुष् + काम + अर्थ + सिद्धये
    \end{itemize}

    \columnbreak
    
    \vspace{10pt}
    \textenglish{
        \noindent Sage Narada said:\\
        We prostrate in front of Lord Vinayaka,\\
        the son of Devi Gauri,\\
        the abode of His devotees,\\
        One who is always remembered for the attainment of longevity, the fulfillment of desires, and wealth.
    }

\end{multicols}

\begin{Verse}
    \setstretch{1.5} % Adjust line spacing within the verse
    प्रथमं वक्रतुण्डं च एकदन्तं द्वितीयकम् ।\\
    तृतीयं कृष्णपिङ्गाक्षं गजवक्त्रं चतुर्थकम् ॥२॥
\end{Verse}


\begin{multicols}{2}
    \setlength{\columnseprule}{0.4pt}
    \begin{itemize}
        \item वक्रतुण्डं = वक्रतुण्डं = वक्र + तुण्डं
        \item एकदन्तं = एक + दन्तं
        \item कृष्णपिङ्गाक्षं = कृष्ण + पिङ्ग + अक्षं
        \item गजवक्त्रं = गज + वक्त्रं
    \end{itemize}

    \columnbreak
    
    \vspace{10pt}
    \textenglish{
        \noindent The twelve names of Lord Ganesha:\\
        First, one with a curved trunk,\\
        Second, one with a single tusk,\\
        Third, one with dark and reddish-brown eyes,\\
        Fourth, one with the face of an elephant
    }

\end{multicols}


\end{document}